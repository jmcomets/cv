\documentclass[10pt, a4paper]{moderncv}

\moderncvtheme[blue]{classic}

\usepackage[utf8]{inputenc}
\usepackage[scale=0.8]{geometry}

\firstname{Jean-Marie}
\familyname{Comets}
\title{Élève ingénieur / Développeur}
\address{20 avenue Albert Einstein}{69100 Villeurbanne}
\mobile{06 99 65 94 07}
\email{jean.marie.comets@gmail.com}
\homepage{jmcomets.com}

\begin{document}
\maketitle

\section{Diplômes et Études}

\cventry{2012 -- actuel}{Formation d'ingénieur -- Informatique}{INSA de
    Lyon}{}{}{}
\cventry{2010 -- 2012}{Premier cycle (classe préparatoire)}{INSA de Lyon}{}{}{}
\cventry{2010}{Bac S avec mention}{}{}{}{}

\section{Expériences}

\cventry{Mai -- Août 2014}{Développeur Ruby On
    Rails}{Fidbacks}{Paris}{France}{Intégration au sein d'une start-up
    collectant des avis clients pour les sites de e-commerce (plateforme
    Prestashop). Développement de l'application web principale déployé sur
    Heroku, ainsi qu'un CRM interne déployé sur Amazon AWS-EC2.}

\cventry{Juin -- Septembre 2013}{Développeur C++}{LIRIS --
    CNRS}{Villeurbanne}{France}{Implémentation d'un diagramme d'épaisseur basé
    sur l'arrangement exact de boules (3D). Codage d'une interface
    Qt/QGLViewer/GLUT, utilisation intensive d'outils d'aide à la compilation
    (CMake, QMake).}

%\cventry{Septembre 2012 -- actuel}{Association JeuxVidéo@INSA}{}{}{}{Ancien
    %secrétaire général, actuel président. L'association a pour but de
    %promouvoir la réalisation de jeux-vidéo et organise des rencontres avec des
    %professionnels du jeu vidéo (conférences avec Ubisoft notamment), ainsi que
    %des évènements de développement.}

%\cventry{Septembre 2012 -- actuel}{Association CinéClub INSA}{}{}{}{Actuel
    %secrétaire général de l'association CinéClub INSA.}

\cventry{Juillet -- Août 2012}{Assistant à la saisie/Aide administrateur
    réseau}{ITEC Services}{Cenon}{France}{Saisie de réponses de patients
    diabètiques dans le cadre d'une étude. Écriture de scripts de gestion des
    machines sur le réseau de l'entreprise.}

\section{Projets}

\cventry{Février -- Avril 2014}{twitto-feels}{}{}{}{Projet scolaire.
    Analyse de sentiments sur une période donnée (Python -- NLTK), en utilisant
    l'API de streaming Twitter.}

\cventry{Décembre -- Février 2014}{KHome}{}{}{}{Projet scolaire.
    Système modulaire pour "maison intelligente", intégrant un magasin
    d'applications en utilisant un protocole de communication réseau
    ressemblant à JSON-RPC.}

\cventry{Janvier 2014}{MindBlast}{}{}{}{Application web proposant des
    suggestions de produits aux commerciaux d'AXA, s'intégrant au sein de
    Salesforce. 4ème place au cours d'un hackaton.}

\cventry{Septembre 2013}{TESS}{}{}{}{Moteur de recherche basé sur l'indexion de
    modèles (Scikit-learn, Elastic Search, Flask). 1ère place au cours d'un
    hackaton organisé par \href{http://www.fhacktory.com}{FHacktory}.}

\section{Informatique}

\cvcomputer{Web}{Python (Django, Flask), JavaScript (\textbf{AngularJS}), Ruby on Rails, PHP}
           {Base de données}{MySQL, \textbf{PostgreSQL}, SQLite, MongoDB, Oracle}
\cvcomputer{POO}{C++, \textbf{Python}, Ruby, Java, JavaScript}
           {Impératif}{C, Pascal}
\cvcomputer{Outils}{\textbf{Git}, SVN, QMake, CMake}
           {Divers}{UML, \textbf{Vim}, Shell, \LaTeX, Linux}

\section{Langues}

\cvlanguage{Anglais}{langue maternelle}{}
\cvlanguage{Français}{bilingue}{}
\cvlanguage{Espagnol}{courant}{}

%\section{Centres d'intérêt}

%\cvline{Loisirs}{cinéma, rugby, algorithmie}

\end{document}
