\documentclass[10pt, a4paper]{moderncv}

\moderncvtheme[red]{classic}

\usepackage[utf8]{inputenc}
\usepackage[scale=0.8]{geometry}


\firstname{Jean-Marie}
\familyname{Comets}
\title{Élève ingénieur / Développeur}
\address{13 rue Spréafico}{69100 Villeurbanne}
\mobile{06 99 65 94 07}
\email{jean.marie.comets@gmail.com}
\homepage{jmcomets.com}

\begin{document}
\maketitle

\section{Diplômes et Études}
\cventry{2012-actuel}{Formation d'ingénieur -- Informatique}{INSA de Lyon}{}{}{}
\cventry{2010-2012}{Premier cycle (classe préparatoire)}{INSA de Lyon}{}{}{}
\cventry{2010}{Bac S avec mention}{}{}{}{}

\section{Expériences}
\cventry{Juin à\\Septembre 2013}{Développeur C++}{LIRIS -- CNRS}{Villeurbanne}{France}{Implémentation d'un diagramme d'épaisseur basé sur l'arrangement exact de boules (3D). Codage d'une interface Qt/QGLViewer/GLUT, utilisation intensive d'outils d'aide à la compilation (CMake, QMake).}
\cventry{Septembre 2012 -- actuel}{Association JeuxVidéo@INSA}{}{}{}{Ancien secrétaire général, actuel président. L'association a pour but de promouvoir la réalisation de jeux-vidéo et organise des rencontres avec des professionnels du jeu vidéo (conférences avec Ubisoft notamment), ainsi que des évènements de développement.}
\cventry{Septembre 2012 -- actuel}{Association CinéClub INSA}{}{}{}{Actuel secrétaire général de l'association CinéClub INSA.}
\cventry{Juillet à\\Août 2012}{Assistant à la saisie/Aide administrateur réseau}{ITEC Services}{Cenon}{France}{Saisie de réponses de patients diabètiques dans le cadre d'une étude.\\Écriture de scripts de gestion des machines sur le réseau de l'entreprise.}

\section{Informatique}
\cvcomputer{Web}{Python (Django, Flask), JavaScript (serveur: ExpressJS; client: JQuery, AngularJS), Ruby (notions en Rails), PHP}{Base de données}{Oracle, MySQL, \\SQLite, MongoDB}
\cvcomputer{POO}{C++, Python, Java, Node.js}{Impératif}{C}
\cvcomputer{Gestionnaire de versions}{Git, SVN}{}{}
\cvcomputer{Conception}{UML}{Divers}{Vim, Shell, Linux, \\notions de \LaTeX.}

\section{Langues}
\cvlanguage{Anglais}{bilingue}{}
\cvlanguage{Espagnol}{courant}{}

\section{Centres d'intérêt}
\cvline{Loisirs}{cinéma, rugby, algorithmique}

\end{document}
